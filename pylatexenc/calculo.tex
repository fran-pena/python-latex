\documentclass{article}
\usepackage{amsmath}
\usepackage{algorithm}
\usepackage{algorithmic}

\begin{document}

\section{Cálculo de raíces}

\begin{equation}
f(x) = \sqrt{x - 2}
\label{eq:calc:f}
\end{equation}

\begin{equation}
a = 0
\label{eq:calc:a}
\end{equation}

\begin{equation}
b = 3
\label{eq:calc:b}
\end{equation}

\begin{algorithm}
\caption{Método de dicotomía}
\label{alg:calc:dicotomia}
\begin{algorithmic}
\STATE tol = 1e-6
\WHILE{b - a > tol}
    \STATE c = (a + b)/2
    \IF{f(c) > 0}
        \STATE b = c
    \ELSE
        \STATE a = c
    \ENDIF
\ENDWHILE
\RETURN c
\end{algorithmic}
\end{algorithm}

\end{document}
